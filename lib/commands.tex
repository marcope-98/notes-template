% ==============================================================
% File     : lib/commands.tex
% Date     : 16 Oct 2022
% Creator  : Marco Peressutti
% ==============================================================

\defbibheading{bibempty}{}
% ==============================================================
% NEW-COMMANDS:
% ==============================================================

% \M         : matrix with square    delimiters/brackets
% \B         : matrix with curly     delimiters/brackets
% \und       : underline math expression in math mode
% \pd        : (first  order) partial derivative
% \td        : (first  order) total   derivative
% \pdd       : (second order) partial derivative 
% \tdd       : (second order) total   derivative
% \omissis   : three dots surrounded by square brackets
% \pexp      : pre exponent (#1) of symbol (#2) where \pexp{#1}{#2} 
% \ret       : left arrow symbol as exponent 
% \smalltodo : [internal] DO NOT USE 
% \side      : size notes (yellow line to the side of the page with text)
% \pside     : phantom side (used in environment where the \side causes compilation errors)
% \ceil      : ceil  brackets
% \floor     : floor brackets 
% \degr      : degree symbol (just an exponent with circle in front of the number)

\newcommand{\M}[1]{\begin{bmatrix}#1\end{bmatrix}}
\newcommand{\und}[1]{\underline{#1}}
\newcommand{\B}[1]{\begin{Bmatrix}#1\end{Bmatrix}}
\newcommand{\pd}[2]{\cfrac{\partial#1}{\partial#2}}
\newcommand{\td}[2]{\cfrac{d#1}{d#2}}
\newcommand{\pdd}[2]{\cfrac{\partial^2#1}{\partial#2^2}}
\newcommand{\tdd}[2]{\cfrac{d^2#1}{d#2^2}}
\newcommand{\omissis}{[\textellipsis\unkern]}
\newcommand{\pexp}[2]{\prescript{#1}{}{#2}{}{}}
\newcommand{\ret}[1]{{#1}^{\leftarrow}}
\newcommand{\smalltodo}[2][]{\todo[caption={#2}, #1, backgroundcolor=white!20!white, bordercolor=white]{\begin{spacing}{0.5}\texttt{#2}\end{spacing}}}
\newcommand{\side}[1]{\smalltodo[size=\footnotesize]{#1}\textbf{#1}}
\newcommand{\pside}[1]{\smalltodo[size=\footnotesize]{#1}}
\newcommand{\ceil}[1]{\left\lceil #1 \right\rceil}
\newcommand{\floor}[1]{\left\lfloor #1 \right\rfloor}
\newcommand{\degr}[1]{^{\circ\!#1}}

% ==============================================================
% RE-NEW-COMMANDS:
% ==============================================================

% prettier \theta symbol
\renewcommand{\theta}{\vartheta}
% prettier \epsilon symbol
\renewcommand{\epsilon}{\varepsilon}
% I have no clue what \arraystretch does prolly used internally in itemize/enumerate environment
\renewcommand{\arraystretch}{1.3}

% ==============================================================
% DECLARES:
% ==============================================================

% \abs      : absolute value operator/delimiters (i.e. |  expression  |)
% \norma    : norm           operator/delimiters (i.e. || expression ||)
% \minimize : "minimize" that allows to write things just under the symbol 
% \argmin   : "argmin" that allows to write things just under the symbol
\DeclarePairedDelimiter\abs{\lvert}{\rvert}
\DeclarePairedDelimiter{\norma}{\lVert}{\rVert}
\DeclareMathOperator*{\minimize}{minimize}
\DeclareMathOperator*{\argmin}{argmin}


% ==============================================================
% MISC:
% ==============================================================

% more compact representtation of itemize
\setitemize{noitemsep,topsep=10pt,parsep=0pt,partopsep=0pt}

% colors
\definecolor{codegreen}{rgb}{0,0.6,0}
\definecolor{codegray}{rgb}{0.5,0.5,0.5}
\definecolor{codepurple}{rgb}{0.58,0,0.82}
\definecolor{backcolour}{rgb}{0.95,0.95,0.92}
\definecolor{myb}{RGB}{45, 111, 177}
\definecolor{myr}{RGB}{199, 68, 64}


% Code snippets style
\lstdefinestyle{mystyle}{
    backgroundcolor=\color{backcolour},   
    commentstyle=\color{codegreen},
    keywordstyle=\color{magenta},
    numberstyle=\tiny\color{codegray},
    stringstyle=\color{codepurple},
    basicstyle=\ttfamily\footnotesize,
    breakatwhitespace=false,         
    breaklines=true,                 
    captionpos=b,                    
    keepspaces=true,                 
    numbers=left,                    
    numbersep=5pt,                  
    showspaces=false,                
    showstringspaces=false,
    showtabs=false,                  
    tabsize=2
}
\lstset{style=mystyle}

\makeatletter
\newcommand\incircbin
{%
  \mathpalette\@incircbin
}
\newcommand\@incircbin[2]
{%
  \mathbin%
  {%
    \ooalign{\hidewidth$#1#2$\hidewidth\crcr$#1\bigcirc$}%
  }%
}
\newcommand{\ooplus}{\incircbin{+}}     % A circle with a plus  inside
\newcommand{\oominus}{\incircbin{-}}    % A circle with a minus inside
\newcommand{\oocirca}{\incircbin{\sim}} % A circle with a tilde inside
\makeatother

\newcommand{\cmark}{\ding{51}} % literally a  tick
\newcommand{\xmark}{\ding{55}} % literally an X

% equal with symbol on top
\newcommand\equal[1]{\stackrel{\mathclap{\footnotesize\mbox{#1}}}{=}}

% text without double lined line
\newcommand{\textbetweendoublerules}[2][.4pt]{%
  \par\addvspace{\topsep}
  \noindent\makebox[\textwidth]{%
    \sbox0{\quad#2\quad}%
    \dimen0=.5\dimexpr\ht0+#1\relax
    \dimen2=-.5\dimexpr\ht0-#1\relax
    \dimen4=.5\dimexpr\textwidth-\wd0\relax
    \setbox2=\vbox to \ht0{%
      \vss
      \hrule width \dimen4 height #1
      \kern 4\dimexpr#1\relax
      \hrule width \dimen4 height #1
      \vss
    }%
    \copy2 \box0 \box2
  }\par\nopagebreak\addvspace{\topsep}%
}


% red box used for examples
\makeatletter
\newtcbtheorem{example}{Example}{enhanced,
	breakable,
	colback=white,
	colframe=myr!80!black,
	attach boxed title to top left={yshift*=-\tcboxedtitleheight},
	fonttitle=\bfseries,
	title={#2},
	boxed title size=title,
	boxed title style={%
			sharp corners,
			rounded corners=northwest,
			colback=tcbcolframe,
			boxrule=0pt,
		},
	underlay boxed title={%
			\path[fill=tcbcolframe] (title.south west)--(title.south east)
			to[out=0, in=180] ([xshift=5mm]title.east)--
			(title.center-|frame.east)
			[rounded corners=\kvtcb@arc] |-
			(frame.north) -| cycle;
		},
	#1
}{def}
\makeatother

% blue box used for definitions
\makeatletter
\newtcbtheorem{definition}{Definition}{enhanced,
	breakable,
	colback=white,
	colframe=myb!80!black,
	attach boxed title to top left={yshift*=-\tcboxedtitleheight},
	fonttitle=\bfseries,
	title={#2},
	boxed title size=title,
	boxed title style={%
			sharp corners,
			rounded corners=northwest,
			colback=tcbcolframe,
			boxrule=0pt,
		},
	underlay boxed title={%
			\path[fill=tcbcolframe] (title.south west)--(title.south east)
			to[out=0, in=180] ([xshift=5mm]title.east)--
			(title.center-|frame.east)examples
		},
	#1
}{def}
\makeatother

% command to use the definition box \define{Name}{Verbose definition of Name}
\newcommand{\define}[2]{\begin{definition}{#1}{}#2\end{definition}}
% command to use the definition box \define{(optional) Example title}{Verbose description of Example}
\newcommand{\eg}[2]{\begin{example}{#1}{}#2\end{example}}


% used to make table of contents for each part
\titleclass{\part}{top}
\titleformat{\part}
  {\centering\normalfont\Huge\bfseries}{}{0pt}{}
\setcounter{secnumdepth}{5}